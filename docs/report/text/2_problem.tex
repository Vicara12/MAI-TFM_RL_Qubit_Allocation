\chapter{The Problem of Qubit Allocation}

% TODO

\begin{comment}

What I want to do here:
- Brief intro to qc
- Logical and physical qubits
- Distributed quantum computing
- Describe the problem
- Metrics: inter core com, normalization, mention swaps

\end{comment}


Some concepts of quantum computing, necessary to understand the problem at hand, present a different
paradigm from classical computation.
This chapter aims to not only describe the task of qubit allocation but also to provide the
necessary background in quantum computation required to understand the motivation behind it.
For a deeper review refer to~\textcite{nielsen2010quantum}.

\section{Fundamentals of Quantum Computation}

The basic unit of compute in quantum computing is the \emph{qubit}.
A qubit is a quantum system that has two basis states, typically represented as $\ket{0}$ and
$\ket{1}$.
Due to the nature of quantum mechanics, it is also allowed that qubits are in \emph{superposition},
that is, a linear combination of these two basis states.
Thus, an arbitrary single qubit quantum state, which we denote as $\ket{\psi}$, is expressed as
\begin{align} \label{eq:single_q_state}
    \ket{\psi} = a \ket{0} + b \ket{1},
\end{align}
with $a,b \in  \mathbb{C}$.

Quantum states are typically formed by several qubits.
In this case, the new basis is formed by all possible combinations of the basis states of the
individual qubits.
For a three qubit quantum state $\ket{\phi}$ we have
\begin{align} \label{eq:two_q_state}
    \ket{\phi} = a_0 \ket{000} + a_1 \ket{001} + a_2 \ket{010} + a_3 \ket{011} +
                 a_4 \ket{100} + a_5 \ket{101} + a_6 \ket{110} + a_7 \ket{111},
\end{align}
and analogously to the single qubit instance, the coefficients $a_i$ form a complex vector in
$\mathbb{C}^8$.

The coefficients that form a quantum state can be altered through quantum operators (also known as
quantum gates), which can act on an arbitrary number of qubits.
An example of a single qubit operator is NOT, which transforms $\ket{0}$ into $\ket{1}$ and
vice-versa. Applying this gate to the state from Eq.~\ref{eq:single_q_state} yields
\begin{align}
    \text{NOT}(\ket\psi) = \text{NOT}(a \ket{0} + b \ket{1}) = a \ket{1} + b \ket{0}.
\end{align}

Another well known quantum operator, which acts on two qubits, is controlled-NOT or CNOT.
This gate is composed of a control and a target qubit: it applies the NOT operator to the target
qubit whenever the control qubit is set to $\ket{1}$.
Taking as an example an arbitrary two-qubit quantum state with the first qubit as control and the
second as target results in
\begin{align}
    \text{CNOT}(\ket{\psi^\prime}) = a_0 \ket{00} + a_1 \ket{01} + a_2 \ket{11} + a_3 \ket{10}.
\end{align}

These two are some examples of quantum gates, but for $n$ qubits any Hermitian
$\mathbb{C}^{2^n} \times \mathbb{C}^{2^n}$ matrix is a valid quantum operator.
This means that there is an infinite number of different quantum gates for any amount of qubits.
However, this does not mean that actual quantum hardware physically applies these operators directly
to its qubits, as this would be overly complex and error-prone.
In practice, quantum compilers decompose large quantum gates into CNOTs (or some other two-qubit gate,
depending on the architecture) and single-qubit gates.
For a detailed explanation of the process, refer to \parencite{shende2005synthesis}.
This is of crucial importance in qubit allocation, because, as we will se later, it limits the
interactions between qubits to be pairwise, greatly simplifying the problem.

A series of quantum gates applied to a set of qubits over time is referred as a~\emph{quantum circuit}.
It is represented via one horizontal line per qubit and vertical symbols (lines,
circles, rectangles, etc) for the operators acting on the qubits.
For example, a circuit in which we apply a NOT to the first qubit, then a CNOT with the second
qubit as a control and the third as a target and then an arbitrary gate H to the third qubit would
look like
\begin{align}
\begin{aligned}
    \begin{quantikz}
        & \targ{} &          &          & \\
        &         & \ctrl{1} &          & \\
        &         & \targ{}  & \gate{H} & \qw
    \end{quantikz}
\end{aligned}.
\end{align}
For the problem of qubit allocation only two-qubit gates are relevant, so from now on we will
exclusively draw these in circuits.
Also, to keep the diagrams architecture-agnostic, we will represent arbitrary two-qubit gates as
\begin{align}
\begin{aligned}
    \begin{quantikz}
        & \ctrl{1} & \\
        & \phase{} & \qw
    \end{quantikz}
\end{aligned},
\end{align}
not to be confused with the controlled phase shift gate.
Note that gates that don't act on the same qubit can be applied simultaneously.
For example, this circuit can be executed in three time steps as
\begin{align}
\begin{aligned}
    \begin{quantikz}
        & \ctrl{1}\slice{} &          & \ctrl{3} \slice{} &          & \\
        & \phase{}         & \ctrl{1} &                   & \ctrl{1} & \\
        & \ctrl{1}         & \phase{} &                   & \phase{} & \\
        & \phase{}         &          & \phase{}          &          & \qw
    \end{quantikz}
\end{aligned}.
\end{align}
Each set of gates that can be executed concurrently in a circuit is known as a \emph{time slice}.


\section{Qubit Allocation}

Analogously to how classical bits are physically implemented as different electric potential levels,
qubits can be materialized using different quantum systems with two or more states, such as
the polarization of light~\parencite{o2007optical}, superconducting
materials~\parencite{huang2020superconducting}, among others.
The actual system that encodes the qubit is known as a~\emph{physical qubit}.
However, when programming quantum computers it is common to use a hardware-independent abstraction
known as~\emph{logical qubit}.

In many paradigms, such as topological quantum computing~\parencite{freedman2003topological}, a logical
qubit is implemented gathering several physical qubits for redundancy and error correction.
However, in this work we consider a one-to-one mapping of physical to logical qubits.